%%%----------------------------------------------------------
\chapter{Actions}
%%%----------------------------------------------------------

This chapter lists all the actions that are usable in the StarCraft environment, which vary per unit.

%%%----------------------------------------------------------
\section{All Units}
The following actions can be executed by any unit.
%%%----------------------------------------------------------

\subsection{cancel/1}
\begin{tabularx}{\textwidth}{lX}
 Description & Cancel the construction or morphing of a unit. \\
 Syntax & \verb|cancel(<TargetId>)| \\
 Example & \verb|cancel(3)| \\
 Parameters & \verb|<TargetId>|: The ID of the unit of which the construction or morphing should be cancelled. \\
 Pre & The targeted unit is incomplete (not fully constructed or morphed). \\
 Post & The targeted unit's construction or morphing will be cancelled; resources will be refunded (and for Zerg the original unit will be restored). \\
 Note & It makes most sense for the \textit{`mapAgent'} to execute this.
\end{tabularx}

\subsection{debugdraw/1}
\begin{tabularx}{\textwidth}{lX}
 Description & Draw text above a unit in the game window. \\
 Syntax & \verb|debugdraw(<Text>)| \\
 Example & \verb|debugdraw("Power Overwhelming")| \\
 Parameters & \verb|<Text>|: The text(string) that should be drawn. \\
 Pre & - \\
 Post & The given text will be drawn above the unit (i.e., it will stay with the unit) in the game window. If the given text is empty, the drawing will be cancelled. \\
 Note & For the \textit{`mapAgent'}, the text will be drawn on a fixed position on the left top of the game window.
\end{tabularx}

\subsection{forfeit/0}
\begin{tabularx}{\textwidth}{lX}
 Description & Forfeit the game. \\
 Syntax & \verb|forfeit| \\
 Pre & The game is in progress. \\
 Post & The game ends with a loss for the player. \\
 Note & It makes most sense for the \textit{`mapAgent'} to execute this.
\end{tabularx}

\subsection{morph/1 (Zerg only)}
\begin{tabularx}{\textwidth}{lX}
 Description & Morph a unit into another unit(type). \\
 Syntax & \verb|morph(<Type>)| \\
 Example & \verb|morph(`Zerg Lurker')| \\
 Parameters & \verb|<Type>|: The type to morph into. See \ref{unittype}. \\
 Pre & The unit is capable of morphing into the given unit type. \\
 Post & The unit's corresponding agent terminates and a new agent is created for the new unit when it is completed (with the same ID).
\end{tabularx}

\newpage

%%%----------------------------------------------------------
\section{Buildings}
The actions in this section can only be executed by buildings (or by some special units that can be loaded or that can produce units of their own).
%%%----------------------------------------------------------

\subsection{buildAddon/1 (Terran only)}
\begin{tabularx}{\textwidth}{lX}
 Description & Build an addon. \\
 Syntax & \verb|buildAddon(<Name>)| \\
 Example & \verb|buildAddon(`Terran Comsat Station')| \\
 Parameters & \verb|<Name>|: The name of the addon that is to be constructed. See \ref{unittype}. \\
 Pre & The building is capable of building the addon and does not already have an addon. \\
 Post & The building starts constructing the addon.
\end{tabularx}

\subsection{cancel/0}
\begin{tabularx}{\textwidth}{lX}
 Description & Cancel the last train or research action. \\
 Syntax & \verb|cancel| \\
 Pre & The unit is training, researching, or constructing an add-on (Terran-only). \\
 Post & The last train, research, or add-on build is cancelled; the resources are refunded.
\end{tabularx}

\subsection{land/2 (Terran only)}
\begin{tabularx}{\textwidth}{lX}
 Description & Land a lifted building on a given location. \\
 Syntax & \verb|land(<X>,<Y>)| \\
 Example & \verb|land(22, 33)| \\
 Parameters & \verb|<X>|: The x-coordinate of the chosen landing location. \\
            & \verb|<Y>|: The y-coordinate of the chosen landing location. \\
 Pre & The unit is currently lifted and the landing location is visible, not obstructed, and fitting for the building. \\
 Post & The unit moves to (if needed) and lands on the chosen location. It reconnects with any addon if applicable.
\end{tabularx}

\subsection{lift/0 (Terran only)}
\begin{tabularx}{\textwidth}{lX}
 Description & Lift a building into the air. \\
 Syntax & \verb|lift| \\
 Pre & The unit is capable of lifting and is not currently performing any other action. \\
 Post & The building lifts into the air.
\end{tabularx}

\subsection{load/2}
\begin{tabularx}{\textwidth}{lX}
 Description & Load a given unit into the unit. \\
 Syntax & \verb|load(<Id>)| \\
 Example & \verb|load(2)| \\
 Parameters & \verb|<Id>|: The ID of the unit to load into this unit. \\
 Pre & The unit is capable of loading the targeted unit and has enough space provided for the targeted unit. \\
 Post & The targeted unit moves towards to the loadable unit and loads into it.
\end{tabularx}

\subsection{research/1}
\begin{tabularx}{\textwidth}{lX}
 Description & Research a tech or upgrade. \\
 Syntax & \verb|research(<Type>)| \\
 Example & \verb|research(`Cloaking Field')| \\
 Parameters & \verb|<Type>|: The name of the tech or upgrade. See \ref{upgradetype} and \ref{techtype}. \\
 Pre & The unit is capable of researching the given tech or upgrade. \\
 Post & The unit starts researching the given tech or upgrade. \\
 Note & The level of an upgrade type (if applicable) is optional; this stacks automatically.
\end{tabularx}

\subsection{train/1}
\begin{tabularx}{\textwidth}{lX}
 Description & Train a unit. \\
 Syntax & \verb|train(<Type>)| \\
 Example & \verb|train(`Protoss Zealot')| \\
 Parameters & \verb|<Type>|: The type of unit to train. See \ref{unittype}. \\
 Pre & The unit is capable of producing the given unit. \\
 Post & The unit starts producing the given unit.
\end{tabularx}

\subsection{unload/1}
\begin{tabularx}{\textwidth}{lX}
 Description & Unload a loaded unit from the unit. \\
 Syntax & \verb|unload(<Id>)| \\
 Example & \verb|unload(3)| \\
 Parameters & \verb|<Id>|: The ID of the unit to unload from this unit.\\
 Pre & The given unit is currently loaded into the unit. \\
 Post & The targeted unit is unloaded from the unit.
\end{tabularx}

\subsection{unloadAll/0}
\begin{tabularx}{\textwidth}{lX}
 Description & Unload all loaded units from the unit. \\
 Syntax & \verb|unloadAll| \\
 Pre & There are units currently loaded into the unit. \\
 Post & All loaded units are unloaded from the unit.
\end{tabularx}

%%%----------------------------------------------------------
\section{Moving Units}
The action in this section can only be executed by moving units (i.e. non-buildings or lifted Terran buildings).
%%%----------------------------------------------------------

\subsection{ability/1}
\begin{tabularx}{\textwidth}{lX}
 Description & Use a (researched) ability. \\
 Syntax & \verb|ability(<Type>)| \\
 Example & \verb|ability(`Burrowing')| \\
 Parameters & \verb|<Type>|: The type of technology to use. See \ref{techtype}. \\
 Pre & The given TechType is researched and the unit is capable of performing the ability (without a target unit or location). \\
 Post & The unit performs the ability. \\
 Note & Behaviour that can be toggled on and off (e.g. Burrow/Cloak/Siege) is also executed by using this action (i.e. once for enabling and then again for disabling).
\end{tabularx}

\subsection{ability/2}
\begin{tabularx}{\textwidth}{lX}
 Description & Use a (researched) ability on a target unit. \\
 Syntax & \verb|ability(<Type>,<Target>)| \\
 Example & \verb|ability(`Archon Warp', 3)| \\
 Parameters & \verb|<Type>|: The type of technology to use. See \ref{techtype}. \\
            & \verb|<Target>|: The target to use the technology on.\\
 Pre & The given TechType is researched, the unit is capable of performing the ability (with some target unit), and the target unit is visible \\
 Post & The unit performs the ability on the target unit.
\end{tabularx}

\subsection{ability/3}
\begin{tabularx}{\textwidth}{lX}
 Description & Use a (researched) ability on a location. \\
 Syntax & \verb|ability(<Type>, <X>, <Y>)| \\
 Example & \verb|ability(`Dark Swarm', 11, 8)| \\
 Parameters & \verb|<Type>|: The type of technology to use. See \ref{techtype}. \\
            & \verb|<X>|: The x-coordinate of the chosen location. \\
            & \verb|<Y>|: The y-coordinate of the chosen location. \\
 Pre & The chosen TechType is researched, the unit is capable of performing the ability (with some target location), and the location is visible. \\
 Post & The unit performs the ability on the given location.
\end{tabularx}

\subsection{attack/1}
\begin{tabularx}{\textwidth}{lX}
 Description & Attack a given unit. \\
 Syntax & \verb|attack(<TargetId>)| \\
 Example & \verb|attack(12)| \\
 Parameters & \verb|<TargetId>|: The ID of the unit that should be attacked. \\
 Pre & The unit is attack capable and the targeted unit is visible and reachable. \\
 Post & The targeted unit is being attacked by your unit. The unit will keep moving towards the enemy unit in order to attack it as long as it is visible and alive. \\
 Note & Terran Medics can use this action to heal friendly units; they cannot attack enemies.
\end{tabularx}

\subsection{attack/2}
\begin{tabularx}{\textwidth}{lX}
 Description & Move to a given location and attack everything on the way. \\
 Syntax & \verb|attack(<X>,<Y>)| \\
 Example & \verb|attack(9, 21)| \\
 Parameters & \verb|<X>|: The x-coordinate of the chosen location. \\
            & \verb|<Y>|: The y-coordinate of the chosen location. \\
 Pre & The unit is attack capable. \\
 Post & The unit moves to the chosen location (or as close as it can get) whilst attacking any enemy unit that it encounters along the way; all such enemy units will be chased until they are no longer visible or alive. \\
 Note & Terran Medics will heal any friendly units they encounter.
\end{tabularx}

\subsection{follow/1}
\begin{tabularx}{\textwidth}{lX}
 Description & Follow a given unit. \\
 Syntax & \verb|follow(<given>)| \\
 Example & \verb|follow(5)| \\
 Parameters & \verb|<given>|: The ID of the unit that should be followed. \\
 Pre & The targeted unit is visible. \\
 Post & The unit follows the selected unit; any enemy will be ignored (i.e. the unit will not automatically attack anything).
\end{tabularx}

\subsection{hold/0}
\begin{tabularx}{\textwidth}{lX}
 Description & Hold a position. \\
 Syntax & \verb|hold| \\
 Pre & - \\
 Post & The unit will hold its current position; any enemy will be ignored (i.e. the unit will not automatically attack anything). 
\end{tabularx}

\subsection{move/2}
\begin{tabularx}{\textwidth}{lX}
 Description & Move to a given location. \\
 Syntax & \verb|move(<X>,<Y>)| \\
 Example & \verb|move(19, 1)| \\
 Parameters & \verb|<X>|: The x-coordinate of the chosen location. \\
            &  \verb|<Y>|: The y-coordinate of the chosen location. \\
 Pre & - \\
 Post & The unit moves to the chosen location (or as close as it can get) whilst ignoring any enemy unit along the way (i.e. the unit will not automatically attack anything).
\end{tabularx}

\subsection{patrol/2}
\begin{tabularx}{\textwidth}{lX}
 Description & Patrol between a unit's current position and the given location. \\
 Syntax & \verb|patrol(<X>,<Y>)| \\
 Example & \verb|patrol(7,8)| \\
 Parameters & \verb|<X>|: The x-coordinate of the chosen location. \\
            &  \verb|<Y>|: The y-coordinate of the chosen location. \\
 Pre & - \\
 Post & The unit patrols between its current position and the chosen location (or as close as it can get); any enemy unit that it encounters will be chased until it is no longer visible or alive, after which the unit will return to its patrol route. \\
 Note & Terran Medics will heal any friendly units they encounter.
\end{tabularx}

\subsection{stop/0}
\begin{tabularx}{\textwidth}{lX}
 Description & Stop performing an action. \\
 Syntax & \verb|stop| \\
 Pre & The unit is performing some kind of action. \\
 Post & The unit stops performing its current action.
\end{tabularx}

\newpage

%%%----------------------------------------------------------
\section{Workers}
The actions in this section can only be executed by worker units.
%%%----------------------------------------------------------

\subsection{build/3}
\begin{tabularx}{\textwidth}{lX}
 Description & Build a building on the given location. \\
 Syntax & \verb|build(<Type>,<X>,<Y>)| \\
 Example & \verb|build(`Terran Supply Depot', 24, 6)| \\
 Parameters & \verb|<Type>|: The type of building that should be built. See \ref{unittype}. \\
            & \verb|<X>|: The x-coordinate of the build location \\
            & \verb|<Y>|: The y-coordinate of the build location \\
 Pre & The unit is capable of constructing the chosen building and the build location is visible, not obstructed, and fitting for the given building. \\
 Post & The unit goes moves the build location (if needed) and starts constructing the building there. Zerg Drones will morph (i.e., the drone will be lost), Terran SCVs will be busy constructing for a while, and Protoss Probes will instantiate a warp (i.e., the probe does not have to remain at the build location). See also \textit{cancel/1} and \textit{repair/1}.
\end{tabularx}

\subsection{gather/1}
\begin{tabularx}{\textwidth}{lX}
 Description & Gather a specific resource (minerals or vespene gas building). \\
 Syntax & \verb|gather(<Id>)| \\
 Example & \verb|gather(32)| \\
 Parameters & \verb|<Id>|: The ID of the chosen resource. \\
 Pre & The given resource is visible and reachable. \\
 Post & The unit starts gathering the chosen resource. It automatically moves back and forth between the resource and the closest resource center.
\end{tabularx}

\subsection{repair/1 (Terran only)}
\begin{tabularx}{\textwidth}{lX}
 Description & Repair a unit or complete an unfinished building. \\
 Syntax & \verb|repair(<Id>)| \\
 Example & \verb|repair(17)| \\
 Parameters & \verb|<Id>|: The ID of the unit to repair or of the building to complete construction of. \\
 Pre & The unit is a Terran SCV, has the resources to repair, and the target unit is visible (and reachable) \\
 Post & The SCV moves towards the selected unit (if needed) and repairs it or resumes its construction.
\end{tabularx}